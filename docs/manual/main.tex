%%%%%%%%%%%%%%%%%%%%%%%%%%%%%%%%%%%%%%%%%
% LUMINA-SN Technical Manual
% Based on The Legrand Orange Book Template
%%%%%%%%%%%%%%%%%%%%%%%%%%%%%%%%%%%%%%%%%

\documentclass[11pt,fleqn]{book}

%%%%%%%%%%%%%%%%%%%%%%%%%%%%%%%%%%%%%%%%%
% The Legrand Orange Book (Simplified)
% Structural Definitions File
% Adapted for LUMINA-SN Manual
%%%%%%%%%%%%%%%%%%%%%%%%%%%%%%%%%%%%%%%%%

%----------------------------------------------------------------------------------------
%	VARIOUS REQUIRED PACKAGES AND CONFIGURATIONS
%----------------------------------------------------------------------------------------

\usepackage[top=3cm,bottom=3cm,left=3cm,right=3cm,headsep=10pt,a4paper]{geometry}

\usepackage{graphicx}
\usepackage{xcolor}
\definecolor{ocre}{RGB}{243,102,25}
\definecolor{lightocre}{RGB}{255,230,210}
\definecolor{darkblue}{RGB}{0,51,102}
\definecolor{codebg}{RGB}{245,245,245}
\definecolor{codeframe}{RGB}{200,200,200}

\usepackage{booktabs}
\usepackage{calc}
\usepackage{multicol}
\usepackage{amsmath,amsfonts,amssymb,amsthm}
\usepackage{tikz}
\usetikzlibrary{positioning,calc}
\usepackage{hyperref}
\hypersetup{
    colorlinks=true,
    linkcolor=ocre,
    citecolor=darkblue,
    urlcolor=ocre,
    hidelinks=false
}

\usepackage{listings}
\lstset{
    basicstyle=\ttfamily\small,
    backgroundcolor=\color{codebg},
    frame=single,
    rulecolor=\color{codeframe},
    breaklines=true,
    numbers=left,
    numberstyle=\tiny\color{gray},
    keywordstyle=\color{blue}\bfseries,
    commentstyle=\color{green!50!black},
    stringstyle=\color{red!70!black},
    showstringspaces=false,
    tabsize=4,
    xleftmargin=2em,
    framexleftmargin=1.5em,
    aboveskip=1em,
    belowskip=1em
}

\lstdefinestyle{cstyle}{
    language=C,
    morekeywords={double,int,uint64_t,bool,int64_t,size_t,typedef,struct}
}
\lstdefinestyle{cudastyle}{
    language=C,
    morekeywords={__global__,__device__,__shared__,atomicAdd,blockIdx,threadIdx,blockDim,gridDim,__syncthreads}
}
\lstdefinestyle{pythonstyle}{
    language=Python,
    morekeywords={self,True,False,None,as,with}
}
\lstdefinestyle{bashstyle}{
    language=bash,
    morekeywords={make,nohup,export}
}

\usepackage{enumitem}
\setlist{nolistsep}

\usepackage[backend=biber,style=numeric-comp,sorting=nyt,natbib=true]{biblatex}
\addbibresource{bibliography.bib}

\usepackage{makeidx}
\makeindex

%----------------------------------------------------------------------------------------
%	FONTS
%----------------------------------------------------------------------------------------

\usepackage{lmodern}
\usepackage[T1]{fontenc}

%----------------------------------------------------------------------------------------
%	CHAPTER/SECTION HEADING STYLES
%----------------------------------------------------------------------------------------

\newcommand{\thechapterimage}{}
\newcommand{\chapterimage}[1]{\renewcommand{\thechapterimage}{#1}}

\newcommand{\thechapterimagetrue}{}

\usepackage{fancyhdr}
\pagestyle{fancy}
\fancyhf{}
\fancyhead[LE,RO]{\sffamily\small\thepage}
\fancyhead[LO]{\sffamily\small\rightmark}
\fancyhead[RE]{\sffamily\small\leftmark}
\renewcommand{\headrulewidth}{0.5pt}
\renewcommand{\footrulewidth}{0pt}
\fancypagestyle{plain}{\fancyhead{}\renewcommand{\headrulewidth}{0pt}}

\usepackage[explicit]{titlesec}

% Numbered chapters
\titleformat{\chapter}[display]
{\normalfont\huge\bfseries\sffamily\color{ocre}}
{\flushright\fontsize{70}{80}\selectfont\color{ocre!30}\thechapter}
{-2em}
{\flushright #1}
[\vspace{-1em}\rule{\textwidth}{2pt}]

% Sections
\titleformat{\section}
{\normalfont\Large\bfseries\sffamily\color{ocre}}
{\thesection}{0.5em}{#1}

\titleformat{\subsection}
{\normalfont\large\bfseries\sffamily}
{\thesubsection}{0.5em}{#1}

\titleformat{\subsubsection}
{\normalfont\normalsize\bfseries\sffamily}
{\thesubsubsection}{0.5em}{#1}

%----------------------------------------------------------------------------------------
%	PART HEADING STYLE
%----------------------------------------------------------------------------------------

\titleformat{\part}[display]
{\centering\normalfont\huge\bfseries\sffamily\color{ocre}}
{\fontsize{80}{90}\selectfont\color{ocre!30}\thepart}
{1em}
{\Huge#1}

%----------------------------------------------------------------------------------------
%	THEOREM/DEFINITION/REMARK ENVIRONMENTS
%----------------------------------------------------------------------------------------

\makeatletter
\newtheoremstyle{ocrenum}
{7pt}{7pt}{\normalfont}{}{}{}{0.25em}
{\small\sffamily\bfseries\color{ocre}\thmname{#1}~\thmnumber{\@ifnotempty{#1}{}\@upn{#2}}%
\thmnote{\ \the\thm@notefont\sffamily\bfseries\color{black}--- #3.\hspace{0.25em}}}

\theoremstyle{ocrenum}
\newtheorem{definitionT}{Definition}[chapter]
\newtheorem{theoremT}{Theorem}[chapter]

\newtheoremstyle{blacknumbox}
{7pt}{7pt}{\normalfont}{}{}{}{0.25em}
{\small\sffamily\bfseries\thmname{#1}~\thmnumber{\@ifnotempty{#1}{}\@upn{#2}}%
\thmnote{\ \the\thm@notefont\sffamily\bfseries--- #3.\hspace{0.25em}}}

\theoremstyle{blacknumbox}
\newtheorem{exampleT}{Example}[chapter]
\newtheorem{remarkT}{Remark}[chapter]
\newtheorem{notationT}{Notation}[chapter]
\makeatother

%----------------------------------------------------------------------------------------
%	COLORED ENVIRONMENTS
%----------------------------------------------------------------------------------------

\usepackage[framemethod=default]{mdframed}

\newmdenv[
    skipabove=7pt,
    skipbelow=7pt,
    rightline=false,
    leftline=true,
    topline=false,
    bottomline=false,
    backgroundcolor=ocre!10,
    linecolor=ocre,
    innerleftmargin=5pt,
    innerrightmargin=5pt,
    innertopmargin=5pt,
    innerbottommargin=5pt,
    leftmargin=0cm,
    rightmargin=0cm,
    linewidth=4pt
]{ocrebox}

\newmdenv[
    skipabove=7pt,
    skipbelow=7pt,
    rightline=false,
    leftline=true,
    topline=false,
    bottomline=false,
    backgroundcolor=darkblue!5,
    linecolor=darkblue,
    innerleftmargin=5pt,
    innerrightmargin=5pt,
    innertopmargin=5pt,
    innerbottommargin=5pt,
    leftmargin=0cm,
    rightmargin=0cm,
    linewidth=4pt
]{bluebox}

\newmdenv[
    skipabove=7pt,
    skipbelow=7pt,
    rightline=false,
    leftline=true,
    topline=false,
    bottomline=false,
    backgroundcolor=green!5,
    linecolor=green!50!black,
    innerleftmargin=5pt,
    innerrightmargin=5pt,
    innertopmargin=5pt,
    innerbottommargin=5pt,
    leftmargin=0cm,
    rightmargin=0cm,
    linewidth=4pt
]{greenbox}

% User-friendly environments
\newenvironment{definition}[1][]{\begin{ocrebox}\begin{definitionT}[#1]}{\end{definitionT}\end{ocrebox}}
\newenvironment{theorem}[1][]{\begin{ocrebox}\begin{theoremT}[#1]}{\end{theoremT}\end{ocrebox}}
\newenvironment{example}[1][]{\begin{greenbox}\begin{exampleT}[#1]}{\end{exampleT}\end{greenbox}}
\newenvironment{remark}[1][]{\begin{bluebox}\begin{remarkT}[#1]}{\end{remarkT}\end{bluebox}}
\newenvironment{notation}[1][]{\begin{bluebox}\begin{notationT}[#1]}{\end{notationT}\end{bluebox}}

% Important box
\newenvironment{important}{%
\begin{mdframed}[
    skipabove=7pt,skipbelow=7pt,
    backgroundcolor=red!5,linecolor=red!70!black,
    leftline=true,topline=false,bottomline=false,rightline=false,
    linewidth=4pt,innerleftmargin=5pt,innerrightmargin=5pt,
    innertopmargin=5pt,innerbottommargin=5pt
]
\textbf{\sffamily\color{red!70!black}Important:}\,
}{\end{mdframed}}

%----------------------------------------------------------------------------------------
%	TABLE OF CONTENTS
%----------------------------------------------------------------------------------------

\usepackage{titletoc}

\contentsmargin{0cm}
\titlecontents{chapter}[1.25cm]
{\addvspace{12pt}\large\sffamily\bfseries}
{\color{ocre}\contentslabel[\Large\thecontentslabel]{1.25cm}\color{ocre}}
{\color{ocre}}
{\color{ocre}\normalsize\;\titlerule*[.5pc]{.}\;\thecontentspage}

\titlecontents{section}[1.25cm]
{\addvspace{3pt}\sffamily\bfseries}
{\contentslabel[\thecontentslabel]{1.25cm}}
{}
{~\titlerule*[.5pc]{.}~\thecontentspage}

\titlecontents{subsection}[1.25cm]
{\addvspace{1pt}\sffamily\small}
{\contentslabel[\thecontentslabel]{1.25cm}}
{}
{~\titlerule*[.5pc]{.}~\thecontentspage}


\begin{document}

%----------------------------------------------------------------------------------------
%	TITLE PAGE
%----------------------------------------------------------------------------------------

\begingroup
\thispagestyle{empty}
\begin{tikzpicture}[remember picture,overlay]
\coordinate [below=12cm] (midpoint) at (current page.north);
\node at (current page.north west)
{\begin{tikzpicture}[remember picture,overlay]
\fill[ocre!20] (0,0) rectangle (\paperwidth,-\paperheight);
\fill[ocre] (0,0) rectangle (\paperwidth,-4cm);
\fill[darkblue] (0,-4cm) rectangle (\paperwidth,-4.3cm);
\draw[anchor=north] (midpoint) node [fill=white,fill opacity=0.85,text opacity=1,inner sep=1.5cm,rounded corners=3pt]{
\Huge\centering\bfseries\sffamily\parbox[c][][t]{0.85\paperwidth}{\centering
\textcolor{darkblue}{LUMINA-SN}\\[15pt]
{\Large\textcolor{ocre}{Monte Carlo Radiative Transfer for Type Ia Supernovae}}\\[8pt]
{\large A Complete Technical Manual}\\[25pt]
{\large K.~J.~Han}\\[5pt]
{\normalsize February 2026}
}};
\end{tikzpicture}};
\end{tikzpicture}
\vfill
\endgroup

%----------------------------------------------------------------------------------------
%	COPYRIGHT PAGE
%----------------------------------------------------------------------------------------

\newpage
~\vfill
\thispagestyle{empty}

\noindent Copyright \copyright\ 2026 K.~J.~Han\\

\noindent \textsc{LUMINA-SN: Luminosity-driven Monte Carlo Supernova Spectral Synthesizer}\\

\noindent This code implements 1D Monte Carlo radiative transfer in homologously expanding supernova ejecta, following the formalism of \citet{lucy1999a,lucy2002,lucy2003} and the TARDIS code \citep{kerzendorf2014}. Written in C99 with CUDA GPU acceleration.\\

\noindent \textit{Version 2.0 --- February 2026}

%----------------------------------------------------------------------------------------
%	TABLE OF CONTENTS
%----------------------------------------------------------------------------------------

\pagestyle{empty}
\tableofcontents
\cleardoublepage
\pagestyle{fancy}

%========================================================================================
%	PART I: PHYSICS & THEORY
%========================================================================================

\part{Physics \& Theory}

%----------------------------------------------------------------------------------------
%	CHAPTER 1: Introduction
%----------------------------------------------------------------------------------------

\chapter{Introduction to Type Ia Supernovae}\index{Type Ia Supernovae}

\section{What is a Type Ia Supernova?}

A Type Ia supernova (SN~Ia) is the thermonuclear explosion of a carbon-oxygen white dwarf star that has reached a critical mass near the Chandrasekhar limit ($M_\mathrm{Ch} \approx 1.4\,M_\odot$). The explosion completely unbinds the star, leaving no compact remnant, and synthesizes approximately $0.6\,M_\odot$ of radioactive $^{56}$Ni \citep{nomoto1984}.

\begin{important}
SN~Ia are cosmological standard candles: their peak luminosity correlates with their light-curve decline rate (the Phillips relation, \citealt{phillips1993}), enabling precise distance measurements to galaxies. This led to the discovery of the accelerating expansion of the Universe.
\end{important}

\section{The Expanding Ejecta}\index{Ejecta!Homologous Expansion}

After the explosion, the ejecta expand freely into a vacuum. Within hours, the expansion reaches a state of \textbf{homologous expansion}:
\begin{equation}
\boxed{v(r, t) = \frac{r}{t}}
\label{eq:homologous}
\end{equation}
where $r$ is the radial distance from the center and $t$ is the time since explosion. This means velocity maps directly to radius: faster material is farther out.

\begin{definition}[Homologous Expansion]
In homologous expansion, each fluid element moves at constant velocity. The density at any velocity coordinate $v$ evolves as:
\begin{equation}
\rho(v, t) = \rho_0(v) \left(\frac{t_0}{t}\right)^3
\end{equation}
where $\rho_0(v)$ is the density at reference epoch $t_0$, and the $t^{-3}$ factor comes from the 3D volumetric dilution.
\end{definition}

\section{Spectral Features}\index{Spectral Features}

SN~Ia spectra are dominated by \textbf{P~Cygni profiles}: blueshifted absorption troughs paired with redshifted emission peaks. These arise because:

\begin{enumerate}
\item Material approaching the observer (along the line of sight) absorbs photons at a blueshifted wavelength.
\item The surrounding envelope re-emits photons isotropically, producing a net emission component redward of the rest wavelength.
\end{enumerate}

\begin{table}[h]
\centering
\caption{Key spectral features in SN~Ia near maximum light.}
\label{tab:spectral_features}
\begin{tabular}{llll}
\toprule
\textbf{Ion} & \textbf{Rest $\lambda$ (\AA)} & \textbf{Observed Range (\AA)} & \textbf{Diagnostic Value} \\
\midrule
Si~\textsc{ii} & 6355 & 5800--6500 & Expansion velocity, temperature \\
Si~\textsc{ii} & 5972 & 5600--6000 & Temperature indicator \\
S~\textsc{ii}  & 5454, 5640 & 5200--5700 & ``W'' feature, burning completeness \\
Ca~\textsc{ii} & 3934, 3968 & 3600--4000 & H\&K lines, high-velocity features \\
Ca~\textsc{ii} & 8498, 8542, 8662 & 8000--8800 & IR triplet \\
Fe~\textsc{ii} & 4500--5200 & 4300--5200 & Iron-group blanketing \\
O~\textsc{i}   & 7774 & 7400--7900 & Unburned oxygen indicator \\
\bottomrule
\end{tabular}
\end{table}

\section{The Inverse Problem}

Given an observed spectrum, we want to determine the physical parameters of the explosion:
\begin{itemize}
\item Luminosity $L$ and photospheric temperature $T_\mathrm{inner}$
\item Density profile $\rho(v)$ and its power-law exponents
\item Chemical composition as a function of velocity (abundance tomography)
\item Time since explosion $t_\mathrm{exp}$
\end{itemize}

LUMINA-SN solves the \emph{forward problem}: given these parameters, compute the emergent spectrum. Combined with Bayesian inference (Part~III), this enables solving the inverse problem.


%----------------------------------------------------------------------------------------
%	CHAPTER 2: Radiative Transfer Theory
%----------------------------------------------------------------------------------------

\chapter{Radiative Transfer in Expanding Atmospheres}\index{Radiative Transfer}

\section{The Transfer Equation}\index{Transfer Equation}

The specific intensity $I_\nu$ along a ray satisfies:
\begin{equation}
\frac{dI_\nu}{ds} = -\kappa_\nu \, I_\nu + j_\nu
\label{eq:rte}
\end{equation}
where $s$ is the path length, $\kappa_\nu$ is the absorption coefficient (opacity), and $j_\nu$ is the emissivity.

In a supernova atmosphere, three opacity sources contribute:
\begin{enumerate}
\item \textbf{Electron scattering} (Thomson): frequency-independent, $\sigma_T = 6.652 \times 10^{-25}$~cm$^2$
\item \textbf{Line opacity} (Sobolev): resonant absorption in atomic transitions
\item \textbf{Continuum opacity}: bound--free and free--free (negligible in SN~Ia, see \S\ref{sec:bfff})
\end{enumerate}

\section{The Sobolev Approximation}\index{Sobolev Approximation}

In homologous expansion, the velocity gradient $dv/dr = 1/t_\mathrm{exp}$ is constant. A photon sweeps through a line's resonance frequency over a very short distance (the \emph{Sobolev length}). This means line interactions are \emph{local}: each line either absorbs the photon or lets it pass.

\begin{definition}[Sobolev Optical Depth]\index{Sobolev Optical Depth}
The optical depth of a line transition $l \to u$ in the Sobolev approximation is:
\begin{equation}
\boxed{\tau_\mathrm{Sob} = \frac{\pi e^2}{m_e c} \, f_{lu} \, \lambda_0 \, t_\mathrm{exp} \, n_l \left(1 - \frac{g_l \, n_u}{g_u \, n_l}\right)}
\label{eq:tau_sobolev}
\end{equation}
where $f_{lu}$ is the oscillator strength, $\lambda_0$ is the rest wavelength, $n_l$ and $n_u$ are the lower and upper level populations, and $g_l$, $g_u$ are the statistical weights.
\end{definition}

The numerical coefficient is:
\begin{equation}
\frac{\pi e^2}{m_e c} = 0.02654 \;\text{cm}^2\,\text{s}^{-1} \quad\text{(CGS)}
\end{equation}

The \emph{escape probability} from a Sobolev line is:
\begin{equation}
\beta_\mathrm{Sob} = \frac{1 - e^{-\tau_\mathrm{Sob}}}{\tau_\mathrm{Sob}}
\end{equation}

\section{The Dilute Radiation Field}\index{Dilution Factor}

Far from the photosphere, the radiation field is diluted. The mean intensity is:
\begin{equation}
J_\nu = W \, B_\nu(T_\mathrm{rad})
\end{equation}
where $W$ is the \textbf{dilution factor} and $T_\mathrm{rad}$ is the \textbf{radiation temperature}. For a geometrically thin photosphere at radius $R_\mathrm{phot}$:
\begin{equation}
W(r) = \frac{1}{2}\left(1 - \sqrt{1 - \left(\frac{R_\mathrm{phot}}{r}\right)^2}\right)
\label{eq:W_geometric}
\end{equation}

\begin{remark}[Physical Interpretation of $W$]
At $r = R_\mathrm{phot}$: $W = 0.5$ (hemisphere illuminated). At $r \gg R_\mathrm{phot}$: $W \approx R_\mathrm{phot}^2 / (4r^2) \to 0$ (point source). In practice, Monte Carlo estimators yield $W$ values that include the effects of line scattering and fluorescence.
\end{remark}

\section{Frame Transformations}\index{Doppler!Frame Transformations}

In homologous expansion, the comoving-frame (CMF) frequency differs from the lab-frame frequency:
\begin{equation}
\nu_\mathrm{cmf} = \nu_\mathrm{lab} \, \left(1 - \frac{\mu \, v}{c}\right) = \nu_\mathrm{lab} \, \left(1 - \frac{\mu \, r}{c \, t_\mathrm{exp}}\right)
\label{eq:doppler}
\end{equation}
where $\mu = \cos\theta$ is the direction cosine of the photon with respect to the radial direction, and $v = r/t_\mathrm{exp}$.

\begin{important}
As a photon propagates through a shell, the comoving-frame frequency changes \emph{linearly} with distance $s$ along the ray:
\begin{equation}
\nu_\mathrm{cmf}(s) = \nu_\mathrm{lab} \left(1 - \frac{r_0\,\mu_0 + s}{c\,t_\mathrm{exp}}\right)
\end{equation}
This monotonic frequency sweep is the basis of the Sobolev sweep algorithm (\S\ref{sec:sobolev_sweep}).
\end{important}

\section{Bound--Free and Free--Free Opacity}\label{sec:bfff}\index{Continuum Opacity}

In SN~Ia ejecta, continuum opacities are negligible because:
\begin{itemize}
\item \textbf{Bound--free}: Optical photons (1.5--2.5~eV) are far below the ionization thresholds of the dominant species (Si~\textsc{ii}: 16.35~eV, Fe~\textsc{ii}: 16.19~eV, Ca~\textsc{ii}: 11.87~eV). Only far-UV photons ($< 1000$~\AA) could ionize these ions.
\item \textbf{Free--free}: Electron densities are very low ($n_e \sim 10^9$~cm$^{-3}$ inner, $\sim 10^6$~cm$^{-3}$ outer), giving $\kappa_\mathrm{ff} \sim 10^{-20}$~cm$^{-1}$.
\end{itemize}

The combined continuum optical depth across a shell is $\tau_\mathrm{cont} \sim 10^{-6}$ to $10^{-9}$, confirming that \textbf{line opacity dominates} the spectrum formation in SN~Ia.


%----------------------------------------------------------------------------------------
%	CHAPTER 3: Monte Carlo Methods
%----------------------------------------------------------------------------------------

\chapter{Monte Carlo Photon Transport}\index{Monte Carlo Methods}

\section{The Indivisible Energy Packet Formalism}\index{Energy Packets}

LUMINA follows the \citet{lucy1999a,lucy2002} formalism, where photons are represented as discrete \textbf{energy packets} ($r$-packets) with properties:

\begin{table}[h]
\centering
\caption{Properties of an $r$-packet in LUMINA.}
\begin{tabular}{lll}
\toprule
\textbf{Symbol} & \textbf{Description} & \textbf{Unit} \\
\midrule
$r$ & Radial position & cm \\
$\mu$ & Direction cosine ($\cos\theta$) & -- \\
$\nu$ & Lab-frame frequency & Hz \\
$\varepsilon$ & Packet energy & erg \\
$i_\mathrm{shell}$ & Current shell index & -- \\
\bottomrule
\end{tabular}
\end{table}

All packets carry the same energy:
\begin{equation}
\varepsilon_\mathrm{pkt} = \frac{L_\mathrm{inner} \, \Delta t}{N_\mathrm{packets}}
\end{equation}
where $L_\mathrm{inner}$ is the luminosity at the inner boundary and $\Delta t$ is the simulation time interval.

\section{Packet Initialization}\index{Packet Initialization}

Packets are emitted from the photosphere ($r = R_\mathrm{inner}$) with:

\subsection{Frequency Sampling from the Planck Function}

The emission frequency is sampled from a blackbody at temperature $T_\mathrm{inner}$ using the \citet{bjorkman2001} method:

\begin{enumerate}
\item Draw $\xi_0 \sim U(0,1)$
\item Find $l_\mathrm{min}$ such that $\displaystyle\sum_{i=1}^{l_\mathrm{min}} i^{-4} \geq \frac{\pi^4}{90}\,\xi_0$
\item Draw $\xi_1, \xi_2, \xi_3, \xi_4 \sim U(0,1)$
\item Compute $x = -\ln(\xi_1 \xi_2 \xi_3 \xi_4) / l_\mathrm{min}$
\item Set $\nu = x \, k_B T_\mathrm{inner} / h$
\end{enumerate}

This is exact (no rejection) and samples directly from $B_\nu(T)$.

\subsection{Angular Distribution}

The direction cosine is sampled from a limb-darkened distribution:
\begin{equation}
\mu = \sqrt{\xi}, \quad \xi \sim U(0,1)
\end{equation}
ensuring that more packets are emitted along the normal direction.

\section{The Packet Propagation Loop}\label{sec:propagation}\index{Propagation Loop}

Each packet undergoes a loop until it escapes ($r > R_\mathrm{outer}$) or is reabsorbed ($r < R_\mathrm{inner}$ moving inward):

\begin{ocrebox}
\textbf{Algorithm: Single Packet Loop}
\begin{enumerate}
\item \textbf{Draw random optical depth}: $\tau_\mathrm{event} = -\ln(\xi)$
\item \textbf{Compute distances} to possible events:
  \begin{itemize}
  \item $d_\mathrm{boundary}$: distance to next shell wall
  \item $d_\mathrm{line}$: distance to next Sobolev resonance
  \item $d_\mathrm{e}$: distance to electron scattering ($\tau = n_e \sigma_T d$)
  \end{itemize}
\item \textbf{Select minimum}: $d_\mathrm{min} = \min(d_\mathrm{boundary}, d_\mathrm{line}, d_\mathrm{e})$
\item \textbf{Move packet}: update $r$, $\mu$, accumulate estimators
\item \textbf{Handle interaction}:
  \begin{itemize}
  \item Boundary: change shell, check escape/reabsorption
  \item Line: scatter, downbranch, or activate macro-atom
  \item Electron: Thomson scatter (isotropic re-emission)
  \end{itemize}
\item \textbf{Repeat} from step 1
\end{enumerate}
\end{ocrebox}

\section{Distance Calculations}\index{Distance Calculations}

\subsection{Distance to Shell Boundary}

For a packet at $(r, \mu)$ in shell $[r_\mathrm{in}, r_\mathrm{out}]$:

\textbf{Outward} ($\mu > 0$ or $r > r_\mathrm{in}/\mu$):
\begin{equation}
d_\mathrm{out} = \sqrt{r_\mathrm{out}^2 - r^2(1 - \mu^2)} - r\mu
\end{equation}

\textbf{Inward} ($\mu < 0$ and impact parameter $< r_\mathrm{in}$):
\begin{equation}
d_\mathrm{in} = -r\mu - \sqrt{r_\mathrm{in}^2 - r^2(1 - \mu^2)}
\end{equation}

\subsection{Distance to Electron Scattering}

\begin{equation}
d_e = \frac{\tau_\mathrm{event}}{n_e \, \sigma_T}
\end{equation}

\subsection{Distance to Sobolev Resonance}\label{sec:sobolev_sweep}\index{Sobolev Sweep}

As the packet traverses the shell, the CMF frequency sweeps through the line list. For a line at rest frequency $\nu_\mathrm{line}$:

\begin{equation}
d_\mathrm{line} = \frac{\nu_\mathrm{cmf}(r, \mu) - \nu_\mathrm{line}}{\nu_\mathrm{lab}} \, c \, t_\mathrm{exp}
\label{eq:d_line}
\end{equation}

\begin{important}
The Sobolev sweep must scan from the \emph{entry} frequency to the \emph{exit} frequency of each shell. The original LUMINA code used a fixed $\pm 1\%$ window, which missed lines during thick shell crossings. The corrected algorithm (Task \#067) uses the full Doppler sweep, increasing line interactions from 0.6\% to $>50\%$ of packet steps.
\end{important}

\section{Line Interaction Types}\index{Line Interactions}

When a packet encounters a line, three interaction modes are available:

\subsection{Resonant Scattering (Mode 0)}\index{Line Interactions!Resonant Scattering}
The packet is absorbed and re-emitted at the same line frequency with a new random direction:
\begin{equation}
\mu_\mathrm{new} \sim U(-1, +1), \quad \nu_\mathrm{new} = \nu_\mathrm{line}
\end{equation}

\subsection{Downbranching (Mode 1)}\index{Line Interactions!Downbranching}
The packet is absorbed into the upper level and re-emitted in a \emph{different} line, selected from the downbranching probability distribution:
\begin{equation}
P(\text{emit in line } j) = \frac{A_{u \to l_j}}{\sum_k A_{u \to l_k}}
\end{equation}
This enables fluorescence: a UV photon absorbed in one line can be re-emitted in the optical.

\subsection{Macro-Atom (Mode 2)}\index{Line Interactions!Macro-Atom}\index{Macro-Atom}

The full macro-atom formalism \citep{lucy2002,lucy2003} activates the atom at the upper energy level and follows a Markov chain of internal transitions:

\begin{ocrebox}
\textbf{Algorithm: Macro-Atom Transition Walk}
\begin{enumerate}
\item Start at activation level $l$
\item Look up transition block for level $l$ (list of possible transitions)
\item Draw $\xi \sim U(0,1)$, select transition $k$ from cumulative probabilities
\item If transition type $\geq 0$ (internal): move to destination level, go to step 2
\item If transition type $< 0$ (emission): emit in the associated line, \textbf{exit}
\end{enumerate}
Maximum 500 iterations (safety limit).
\end{ocrebox}

Emission types include:
\begin{itemize}
\item $-1$: Bound--bound emission (emit photon in a specific line)
\item $-2$: Bound--free emission (photoionization, thermalization)
\item $-3$: Free--free emission (thermal)
\item $-4$: Adiabatic cooling
\end{itemize}

\section{Monte Carlo Estimators}\index{Estimators}

As packets propagate, they contribute to estimators that probe the radiation field.

\begin{definition}[Mean Intensity Estimator]\index{Estimators!Mean Intensity}
\begin{equation}
\hat{J}_\nu = \frac{1}{4\pi \, V \, \Delta t} \sum_\mathrm{packets} \varepsilon_\mathrm{cmf} \, \Delta s
\end{equation}
In practice, two scalar estimators are accumulated per shell:
\begin{align}
j &= \sum_\mathrm{packets} \varepsilon_\mathrm{cmf} \, \Delta s \label{eq:j_est} \\
\overline{\nu} &= \frac{\sum_\mathrm{packets} \varepsilon_\mathrm{cmf} \, \nu_\mathrm{cmf} \, \Delta s}{\sum_\mathrm{packets} \varepsilon_\mathrm{cmf} \, \Delta s} \label{eq:nubar_est}
\end{align}
\end{definition}

From these, the radiation temperature and dilution factor are recovered:
\begin{align}
T_\mathrm{rad} &= T_\mathrm{rad,const} \times \frac{\bar{\nu}}{j} \label{eq:Trad_from_est} \\[4pt]
W &= \frac{j}{4\,\sigma_\mathrm{SB}\,T_\mathrm{rad}^4\,\Delta t\,V} \label{eq:W_from_est}
\end{align}
where $T_\mathrm{rad,const} = \frac{\pi^4}{15 \cdot 24 \cdot \zeta(5)} \frac{h}{k_B} = 1.2523 \times 10^{-11}$~K$\cdot$s.

\section{Convergence: The Iteration Loop}\index{Convergence}

LUMINA iterates between transport and plasma calculations:

\begin{enumerate}
\item \textbf{Transport}: Run $N_\mathrm{packets}$ through the ejecta, accumulating estimators.
\item \textbf{Radiation field update}: Compute $T_\mathrm{rad}$, $W$ from estimators, with \textbf{damping}:
\begin{equation}
X_\mathrm{new} = X_\mathrm{old} + d \cdot (X_\mathrm{est} - X_\mathrm{old}), \quad d = 0.5
\end{equation}
\item \textbf{Plasma update}: Recompute ionization, level populations, $\tau_\mathrm{Sob}$.
\item \textbf{$T_\mathrm{inner}$ update} (after hold iterations):
\begin{equation}
\boxed{T_\mathrm{inner,new} = T_\mathrm{inner,old} + d \cdot \left[T_\mathrm{inner} \left(\frac{L_\mathrm{emitted}}{L_\mathrm{requested}}\right)^{-0.5} - T_\mathrm{inner,old}\right]}
\label{eq:t_inner_update}
\end{equation}
\end{enumerate}

\begin{remark}[Exponent $-0.5$ vs $+0.25$]
Na\"ively, Stefan--Boltzmann gives $T \propto L^{0.25}$, suggesting the correction factor should be $(L_\mathrm{em}/L_\mathrm{req})^{0.25}$. However, TARDIS uses the exponent $-0.5$ because changing $T_\mathrm{inner}$ also changes the opacity (through ionization), leading to non-linear feedback. The empirical $-0.5$ accounts for this.
\end{remark}


%========================================================================================
%	PART II: CODE ARCHITECTURE
%========================================================================================

\part{Code Architecture}

%----------------------------------------------------------------------------------------
%	CHAPTER 4: Overview & Build System
%----------------------------------------------------------------------------------------

\chapter{Overview \& Build System}\index{Build System}

\section{File Structure}

LUMINA-SN consists of the following source files:

\begin{table}[h]
\centering
\caption{Source files and their roles.}
\begin{tabular}{lrl}
\toprule
\textbf{File} & \textbf{Lines} & \textbf{Purpose} \\
\midrule
\texttt{lumina.h} & 378 & Master header: all structures, constants, prototypes \\
\texttt{lumina\_transport.c} & 515 & CPU transport kernel \\
\texttt{lumina\_plasma.c} & 524 & Plasma solver \& convergence \\
\texttt{lumina\_atomic.c} & 700 & Atomic data loading (HDF5, CSV, NPY) \\
\texttt{lumina\_main.c} & 466 & Main driver \& iteration loop \\
\texttt{lumina\_cuda.cu} & 1353 & CUDA GPU transport kernel \\
\midrule
\textbf{Total} & \textbf{3936} & \\
\bottomrule
\end{tabular}
\end{table}

\section{Build System}\index{Build System!Makefile}

\begin{lstlisting}[style=bashstyle,caption={Building LUMINA-SN}]
# CPU build (serial)
make

# CPU build with OpenMP parallelism
make OMP=1

# CUDA GPU build
make cuda
\end{lstlisting}

\begin{table}[h]
\centering
\caption{Compiler flags.}
\begin{tabular}{lll}
\toprule
\textbf{Target} & \textbf{Compiler} & \textbf{Flags} \\
\midrule
CPU & \texttt{gcc} & \texttt{-O2 -Wall -Wextra -std=c11 -lm} \\
CPU+OMP & \texttt{gcc} & adds \texttt{-fopenmp} \\
GPU & \texttt{nvcc} & \texttt{-O2 -arch=sm\_89 -std=c++14} \\
\bottomrule
\end{tabular}
\end{table}

\begin{important}
\texttt{make clean} deletes CSV output files. Always save important spectral outputs before rebuilding.
\end{important}

\section{Dependencies}

\begin{itemize}
\item \textbf{HDF5} (optional): For loading atomic data from \texttt{kurucz\_cd23\_chianti\_H\_He.h5}
\item \textbf{CUDA Toolkit $\geq$ 12.0}: For GPU builds (tested with CUDA 13.0, sm\_89)
\item \textbf{OpenMP}: For CPU parallelism (optional)
\item \textbf{Standard C library}: \texttt{math.h}, \texttt{stdio.h}, \texttt{stdlib.h}, \texttt{string.h}
\end{itemize}

\section{Execution}\index{Execution}

\begin{lstlisting}[style=bashstyle,caption={Running LUMINA-SN}]
# CPU: reference model with 200K packets, 20 iterations
./lumina tardis_reference 200000 20

# CUDA: same with GPU acceleration
./lumina_cuda atomic/kurucz_cd23_chianti_H_He.h5 200000 output.csv
\end{lstlisting}


%----------------------------------------------------------------------------------------
%	CHAPTER 5: Data Structures
%----------------------------------------------------------------------------------------

\chapter{Data Structures}\index{Data Structures}

\section{The \texttt{RPacket} Structure}\index{Data Structures!RPacket}

The fundamental unit of the Monte Carlo simulation:

\begin{lstlisting}[style=cstyle,caption={\texttt{RPacket} --- photon energy packet}]
typedef struct {
    double r;                // radial position [cm]
    double mu;               // cos(theta) direction
    double nu;               // lab-frame frequency [Hz]
    double energy;           // packet energy [erg]
    int    current_shell_id; // shell index [0..n_shells-1]
    int    next_line_id;     // Sobolev sweep bookmark
    PacketStatus status;     // IN_PROCESS / EMITTED / REABSORBED
    int    index;            // packet ID (for RNG seeding)
} RPacket;
\end{lstlisting}

\section{Geometry}\index{Data Structures!Geometry}

The 1D spherically symmetric ejecta model:

\begin{lstlisting}[style=cstyle,caption={\texttt{Geometry} --- shell structure}]
typedef struct {
    int    n_shells;         // number of shells (default: 30)
    double *r_inner;         // [n_shells] inner radii [cm]
    double *r_outer;         // [n_shells] outer radii [cm]
    double *v_inner;         // [n_shells] inner velocities [cm/s]
    double *v_outer;         // [n_shells] outer velocities [cm/s]
    double time_explosion;   // t_exp [seconds]
} Geometry;
\end{lstlisting}

The radii are derived from velocities via homologous expansion: $r = v \times t_\mathrm{exp}$.

\section{Opacity State}\index{Data Structures!OpacityState}

Pre-computed opacity data used during transport:

\begin{lstlisting}[style=cstyle,caption={\texttt{OpacityState} --- line and continuum opacities}]
typedef struct {
    int n_lines, n_shells;
    double *line_list_nu;         // [n_lines] rest frequencies, descending
    double *tau_sobolev;          // [n_lines * n_shells]
    double *electron_density;     // [n_shells] n_e [cm^-3]
    double *t_electrons;          // [n_shells] T_e [K]

    // Macro-atom transition data
    int n_macro_levels, n_macro_transitions;
    int    *macro_block_references;       // [n_levels+1]
    int    *transition_type;              // [n_transitions]
    int    *destination_level_id;         // [n_transitions]
    int    *transition_line_id;           // [n_transitions]
    double *transition_probabilities;     // [n_transitions * n_shells]
    int    *line2macro_level_upper;       // [n_lines]
} OpacityState;
\end{lstlisting}

\begin{remark}[Line List Ordering]
Lines are sorted in \emph{descending} frequency order. As a packet's CMF frequency decreases while traversing a shell, it encounters lines from high to low frequency. This ordering enables efficient forward-only scanning.
\end{remark}

\section{Plasma State}\index{Data Structures!PlasmaState}

Thermodynamic state updated each iteration:

\begin{lstlisting}[style=cstyle,caption={\texttt{PlasmaState} --- radiation field quantities}]
typedef struct {
    int    n_shells;
    double *W;               // [n_shells] dilution factor
    double *T_rad;           // [n_shells] radiation temperature [K]
    double *rho;             // [n_shells] mass density [g/cm^3]
    double *n_electron;      // [n_shells] electron density [cm^-3]
    double T_e_T_rad_ratio;  // default: 0.9
} PlasmaState;
\end{lstlisting}

\section{Atomic Data}\index{Data Structures!AtomicData}

Comprehensive atomic physics database for the plasma solver:

\begin{lstlisting}[style=cstyle,caption={\texttt{AtomicData} --- atomic physics for Saha--Boltzmann}]
typedef struct {
    // Per-line data
    int    *line_atomic_number;    // [n_lines] Z
    int    *line_ion_number;       // [n_lines] ionization stage
    double *line_f_lu;             // [n_lines] oscillator strength
    double *line_wavelength_cm;    // [n_lines] rest wavelength

    // Energy levels
    int     n_levels;
    double *level_energy_eV;       // [n_levels]
    int    *level_g;               // [n_levels] statistical weight
    int    *level_metastable;      // [n_levels] 0 or 1

    // Ionization energies
    int     n_ionization;
    double *ioniz_energy_eV;       // [n_ionization]

    // Zeta correction factors (dilute non-LTE)
    double *zeta_data;             // [n_zeta_ions * n_zeta_temps]

    // Abundances
    double *abundances;            // [n_elements * n_shells]

    // Computed quantities (updated each iteration)
    double *ion_number_density;    // [n_ion_pops * n_shells]
    double *partition_functions;   // [n_ion_pops * n_shells]
} AtomicData;
\end{lstlisting}

\section{Monte Carlo Estimators}\index{Data Structures!Estimators}

Accumulated during transport, used to update the radiation field:

\begin{lstlisting}[style=cstyle,caption={\texttt{Estimators} --- radiation field accumulators}]
typedef struct {
    double *j_estimator;         // [n_shells] integral of E*ds
    double *nu_bar_estimator;    // [n_shells] integral of E*nu*ds
    double *j_blue_estimator;    // [n_lines * n_shells] (CPU only)
    double *Edotlu_estimator;    // [n_lines * n_shells] (CPU only)
} Estimators;
\end{lstlisting}

\begin{remark}[GPU Limitation]
The \texttt{j\_blue} and \texttt{Edotlu} estimators are \emph{not computed} on the GPU because they require $n_\mathrm{lines} \times n_\mathrm{shells} \approx 4$~million atomic additions per iteration --- prohibitively expensive for \texttt{atomicAdd}.
\end{remark}


%----------------------------------------------------------------------------------------
%	CHAPTER 6: Transport Engine
%----------------------------------------------------------------------------------------

\chapter{Transport Engine}\index{Transport Engine}

\section{Overview}

The transport engine (\texttt{lumina\_transport.c}, 515 lines) propagates $r$-packets through the ejecta. It is the most performance-critical component.

\section{The \texttt{trace\_packet} Function}\index{Transport Engine!trace\_packet}

This function computes the next interaction event for a packet:

\begin{enumerate}
\item Compute distance to shell boundaries ($d_\mathrm{boundary}$)
\item Scan the Sobolev line list for resonances ($d_\mathrm{line}$, accumulated $\tau$)
\item Compute distance to electron scattering ($d_e = \tau_\mathrm{event} / (n_e \sigma_T)$)
\item Return the minimum distance and interaction type
\end{enumerate}

\section{Sobolev Line Sweep}\index{Transport Engine!Sobolev Sweep}

The sweep algorithm processes lines in descending frequency order:

\begin{lstlisting}[style=cstyle,caption={Sobolev sweep (simplified)}]
double tau_trace_combined = 0.0;
for (int j = pkt->next_line_id; j < n_lines; j++) {
    double nu_line = line_list_nu[j];
    double d_line = compute_d_line(pkt, nu_line);

    if (d_line < 0 || d_line > d_boundary) break;

    double tau_line = tau_sobolev[j * n_shells + shell_id];
    tau_trace_combined += tau_line;

    if (tau_trace_combined > tau_event) {
        // Line interaction!
        *interaction_type = INTERACTION_LINE;
        *d_min = d_line;
        pkt->next_line_id = j;
        return;
    }
}
\end{lstlisting}

\section{Thomson Scattering}\index{Transport Engine!Thomson Scattering}

Elastic scattering with free electrons:

\begin{enumerate}
\item Transform packet energy/frequency to comoving frame at current angle
\item Sample new direction: $\mu_\mathrm{new} \sim U(-1, +1)$ (isotropic in CMF)
\item Transform back to lab frame with new angle
\item Energy is conserved in the comoving frame
\end{enumerate}

\begin{equation}
\varepsilon_\mathrm{lab,new} = \varepsilon_\mathrm{cmf} \times \frac{1}{1 - \mu_\mathrm{new} \cdot v/c}
\end{equation}

\section{Boundary Crossing}\index{Transport Engine!Boundary Crossing}

When a packet crosses a shell boundary:
\begin{enumerate}
\item Update shell index: $i_\mathrm{shell} \leftarrow i_\mathrm{shell} + \delta$ where $\delta = +1$ (outward) or $-1$ (inward)
\item Nudge position by $\epsilon = 10^{-10} \times \Delta r_\mathrm{shell}$ into the new shell
\item Check for escape ($i > n_\mathrm{shells} - 1$) or reabsorption ($i < 0$, inward-moving)
\end{enumerate}

\begin{important}
The position nudge is critical. Without it, the packet lands exactly on the boundary, and the next distance calculation returns $d = 0$, causing an infinite loop (Task \#024).
\end{important}


%----------------------------------------------------------------------------------------
%	CHAPTER 7: Plasma Physics Solver
%----------------------------------------------------------------------------------------

\chapter{Plasma Physics Solver}\index{Plasma Solver}

\section{Overview}

The plasma solver (\texttt{lumina\_plasma.c}, 524 lines) computes the thermodynamic state of the ejecta for each iteration. It implements the TARDIS-compatible nebular approximation \citep{mazzali1993}.

\section{Step 1: Partition Functions}\index{Plasma Solver!Partition Functions}

The partition function for ion $(Z, \text{stage})$ in shell $s$ is:
\begin{equation}
\mathcal{Z}(Z, \text{stage}, s) = \underbrace{\sum_{i \in \text{meta}} g_i \, e^{-E_i / k_B T_\mathrm{rad}(s)}}_{\mathcal{Z}_\mathrm{meta}} + W(s) \cdot \underbrace{\sum_{i \in \text{non-meta}} g_i \, e^{-E_i / k_B T_\mathrm{rad}(s)}}_{\mathcal{Z}_\mathrm{non}}
\label{eq:partition}
\end{equation}

\begin{important}
The Boltzmann factors use $T_\mathrm{rad}$ for \emph{all} levels (both metastable and non-metastable). Earlier code incorrectly used $T_e$ for metastable levels. The dilution factor $W$ suppresses the non-metastable contribution at large distances from the photosphere.
\end{important}

\section{Step 2: Electron Density}\index{Plasma Solver!Electron Density}

Computed iteratively with TARDIS-style damping:

\begin{ocrebox}
\textbf{Algorithm: Electron Density Iteration}
\begin{enumerate}
\item Start with initial guess $n_e^{(0)}$
\item For each element: compute ionization ratios using nebular Saha (Eq.~\ref{eq:saha_neb})
\item Normalize ion populations to element abundance
\item Compute $n_e^{(\text{calc})} = \sum_\text{ions} \text{stage} \times n_\text{ion}$
\item Damped update: $n_e^{(k+1)} = 0.5 \times n_e^{(\text{calc})} + 0.5 \times n_e^{(k)}$
\item Convergence: $|n_e^{(k+1)} - n_e^{(k)}| / n_e^{(k)} < 0.05$
\end{enumerate}
\end{ocrebox}

\section{Step 3: Nebular Saha Ionization}\index{Plasma Solver!Saha Equation}

The ionization ratio between consecutive stages is:
\begin{equation}
\boxed{\frac{n_{i+1}}{n_i} = \frac{\Phi_\mathrm{neb}}{n_e}}
\label{eq:saha_neb}
\end{equation}

where the nebular ionization coefficient $\Phi_\mathrm{neb}$ is:
\begin{equation}
\Phi_\mathrm{neb} = \Phi_\mathrm{LTE} \times W \times \left[\zeta \cdot \delta + W \cdot (1 - \zeta)\right] \times \sqrt{\frac{T_e}{T_\mathrm{rad}}}
\end{equation}

\begin{align}
\Phi_\mathrm{LTE} &= \frac{\mathcal{Z}_{i+1}}{\mathcal{Z}_i} \times 2 \times g_e \times e^{-\chi / k_B T_\mathrm{rad}} \\[4pt]
g_e &= \left(\frac{2\pi m_e k_B T_\mathrm{rad}}{h^2}\right)^{3/2} \\[4pt]
\delta &= \frac{T_e}{T_\mathrm{rad}} \exp\left[\chi \left(\frac{1}{k_B T_\mathrm{rad}} - \frac{1}{k_B T_e}\right)\right]
\end{align}

Here $\chi$ is the ionization energy, and $\zeta$ is a non-LTE correction factor interpolated from tabulated values.

\section{Step 4: $\tau_\mathrm{Sobolev}$ Update}\index{Plasma Solver!Tau Sobolev}

With ion populations known, $\tau_\mathrm{Sob}$ is recomputed for each line and shell using Eq.~\eqref{eq:tau_sobolev}. The level populations follow the Boltzmann distribution within each ion:

\begin{equation}
n_{l,\text{meta}} = \frac{g_l}{\mathcal{Z}} \, n_\text{ion} \, e^{-E_l / k_B T_\mathrm{rad}}, \quad
n_{l,\text{non}} = W \times \frac{g_l}{\mathcal{Z}} \, n_\text{ion} \, e^{-E_l / k_B T_\mathrm{rad}}
\end{equation}

\section{Step 5: Radiation Field Update}\index{Plasma Solver!Radiation Field}

From the MC estimators $j$ and $\overline{\nu}$ (Eqs.~\ref{eq:j_est}--\ref{eq:nubar_est}):
\begin{align}
T_\mathrm{rad,est}(s) &= 1.2523 \times 10^{-11} \times \frac{\overline{\nu}(s)}{j(s)} \quad \text{[K]} \\
W_\mathrm{est}(s) &= \frac{j(s)}{4\,\sigma_\mathrm{SB}\,T_\mathrm{rad}^4(s)\,\Delta t\,V(s)}
\end{align}

All quantities are damped:
\begin{equation}
X_\mathrm{new} = X_\mathrm{old} + 0.5 \times (X_\mathrm{est} - X_\mathrm{old})
\end{equation}


%----------------------------------------------------------------------------------------
%	CHAPTER 8: Atomic Data System
%----------------------------------------------------------------------------------------

\chapter{Atomic Data System}\index{Atomic Data}

\section{Data Sources}

LUMINA uses the TARDIS reference atomic dataset, originally from Kurucz CD23 and CHIANTI:

\begin{table}[h]
\centering
\caption{Atomic data files.}
\label{tab:atomic_files}
\begin{tabular}{lll}
\toprule
\textbf{File} & \textbf{Format} & \textbf{Contents} \\
\midrule
\texttt{line\_list.csv} & CSV & $\nu$, $Z$, ion, $f_{lu}$, $\lambda$ per line \\
\texttt{levels.csv} & CSV & $E$ (eV), $g$, metastable flag per level \\
\texttt{ionization\_energies.csv} & CSV & $\chi$ per ion \\
\texttt{tau\_sobolev.npy} & NPY & $\tau_\mathrm{Sob}$ reference values \\
\texttt{transition\_probabilities.npy} & NPY & Macro-atom transition probs \\
\texttt{macro\_atom\_data.csv} & CSV & Transition types, destinations \\
\texttt{zeta\_data.npy} & NPY & Non-LTE correction factors \\
\texttt{abundances.csv} & CSV & Mass fractions per shell \\
\bottomrule
\end{tabular}
\end{table}

\section{The NPY Format Reader}\index{Atomic Data!NPY Format}

LUMINA includes a custom NPY reader (no NumPy dependency in C):

\begin{enumerate}
\item Read 6-byte magic: \texttt{\textbackslash x93NUMPY}
\item Read version (1 or 2) and header length
\item Parse Python dict header for \texttt{shape}, \texttt{dtype}, \texttt{fortran\_order}
\item Read raw binary data
\item Transpose if Fortran-ordered
\end{enumerate}

\section{The CSV Parser}\index{Atomic Data!CSV Parser}

\begin{important}
The \texttt{macro\_atom\_data.csv} header starts with an unnamed index column: ``\texttt{,atomic\_number,...}''. The standard \texttt{strtok()} function skips leading delimiters, causing a column offset of $-1$. LUMINA uses a manual field-by-field parser that handles empty fields correctly.
\end{important}


%----------------------------------------------------------------------------------------
%	CHAPTER 9: CUDA GPU Implementation
%----------------------------------------------------------------------------------------

\chapter{CUDA GPU Implementation}\index{CUDA}

\section{Design Philosophy}

The GPU implementation maps \textbf{one CUDA thread per packet}. Each thread independently propagates its packet through the ejecta, requiring no inter-thread communication except for atomic estimator updates.

\section{Memory Layout}\index{CUDA!Memory Layout}

\begin{table}[h]
\centering
\caption{GPU memory allocation.}
\begin{tabular}{llr}
\toprule
\textbf{Data} & \textbf{Access} & \textbf{Size (200K pkts)} \\
\midrule
Line frequencies & Read-only & $n_\mathrm{lines} \times 8$~B \\
$\tau_\mathrm{Sob}$ & Read-only & $n_\mathrm{lines} \times n_\mathrm{shells} \times 8$~B \\
Transition probs & Read-only & $n_\mathrm{trans} \times n_\mathrm{shells} \times 8$~B \\
Shell geometry & Read-only & $n_\mathrm{shells} \times 4 \times 8$~B \\
RNG states & Read/write & $N_\mathrm{pkt} \times 4 \times 8$~B \\
$j$, $\overline{\nu}$ estimators & Atomic write & $n_\mathrm{shells} \times 2 \times 8$~B \\
Output arrays & Write-only & $N_\mathrm{pkt} \times 3 \times 8$~B \\
\bottomrule
\end{tabular}
\end{table}

Total GPU memory: approximately 2~GB for a typical run.

\section{Kernel Launch Configuration}\index{CUDA!Kernel Launch}

\begin{lstlisting}[style=cudastyle,caption={Kernel launch}]
int threads_per_block = 256;
int blocks = (n_packets + threads_per_block - 1)
             / threads_per_block;
// Max blocks: 131072 (was 1024 -- critical bug #13)
transport_kernel<<<blocks, threads_per_block>>>(...);
\end{lstlisting}

\begin{important}
The original code had \texttt{CUDA\_MAX\_BLOCKS = 1024}, limiting execution to 262K packets regardless of $N_\mathrm{packets}$. For 2M packets, only 13\% executed, causing $j_\mathrm{estimator}$ to be 76$\times$ too low. This was the single largest GPU bug (Task \#13).
\end{important}

\section{Random Number Generation}\index{CUDA!RNG}

Each thread uses an independent \textbf{xoshiro256**} generator with 256 bits of state (4 $\times$ \texttt{uint64}). Seeds are derived from the packet index via SplitMix64:

\begin{lstlisting}[style=cudastyle,caption={Per-thread RNG initialization}]
__device__ void init_rng(uint64_t *state, uint64_t seed) {
    // SplitMix64 to expand seed into 4 state words
    state[0] = splitmix64(&seed);
    state[1] = splitmix64(&seed);
    state[2] = splitmix64(&seed);
    state[3] = splitmix64(&seed);
}
\end{lstlisting}

\section{Atomic Estimator Updates}\index{CUDA!Atomic Operations}

The $j$ and $\overline{\nu}$ estimators are updated using CUDA \texttt{atomicAdd}:

\begin{lstlisting}[style=cudastyle,caption={Estimator accumulation on GPU}]
__device__ void update_estimators(
    double *d_j_est, double *d_nu_bar_est,
    int shell_id, double comov_energy,
    double comov_nu, double distance)
{
    atomicAdd(&d_j_est[shell_id],
              comov_energy * distance);
    atomicAdd(&d_nu_bar_est[shell_id],
              comov_energy * distance * comov_nu);
}
\end{lstlisting}

Since there are only $n_\mathrm{shells} = 30$ accumulation targets, contention is manageable.

\section{Performance}\index{CUDA!Performance}

\begin{table}[h]
\centering
\caption{CPU vs GPU performance (NVIDIA RTX 5000 Ada, sm\_89).}
\begin{tabular}{lrrrr}
\toprule
\textbf{$N_\mathrm{packets}$} & \textbf{CPU (1 core)} & \textbf{CPU (OMP64)} & \textbf{GPU} & \textbf{Speedup} \\
\midrule
20,000 & 0.7\,s & 0.1\,s & 0.08\,s & $9\times$ \\
200,000 & 7.2\,s & 1.1\,s & 0.73\,s & $10\times$ \\
2,000,000 & 72\,s & 11\,s & 7.3\,s & $10\times$ \\
20,000,000 & 720\,s & 110\,s & 73\,s & $10\times$ \\
\bottomrule
\end{tabular}
\end{table}

\begin{remark}[Statistical Accuracy]
GPU and CPU produce statistically identical results. At 200K packets: $W$ error $1.06\%$, $T_\mathrm{rad}$ error $0.58\%$ (relative to TARDIS reference). The scaling follows $\sigma \propto N^{-0.35}$ to $N^{-0.40}$, close to Poisson ($N^{-0.5}$).
\end{remark}

\section{Resolved GPU Bugs}\index{CUDA!Bug Fixes}

Four critical bugs were identified and fixed during development:

\begin{enumerate}
\item \textbf{MAX\_BLOCKS = 1024}: Only 262K threads could launch. Fixed to 131072.
\item \textbf{Shared memory race}: \texttt{\_\_shared\_\_ ShellCache} shared by all 256 threads, but only thread 0 loaded data. Fixed: use L1-cached global memory.
\item \textbf{Counter accumulation}: Escape/reabsorb counters not reset between iterations. Fixed: explicit reset in \texttt{cuda\_reset\_estimators()}.
\item \textbf{Boundary sticking}: Packets land exactly on shell boundaries due to floating-point precision. Fixed: explicit nudge by $10^{-10} \times \Delta r$.
\end{enumerate}


%========================================================================================
%	PART III: USAGE & APPLICATIONS
%========================================================================================

\part{Usage \& Applications}

%----------------------------------------------------------------------------------------
%	CHAPTER 10: Quick Start
%----------------------------------------------------------------------------------------

\chapter{Installation \& Quick Start}\index{Installation}

\section{Prerequisites}

\begin{table}[h]
\centering
\caption{System requirements.}
\begin{tabular}{ll}
\toprule
\textbf{Component} & \textbf{Requirement} \\
\midrule
C Compiler & GCC $\geq$ 9.0 (C11 support) \\
CUDA (optional) & Toolkit $\geq$ 12.0, compute capability $\geq$ 7.0 \\
HDF5 (optional) & \texttt{libhdf5-dev} for atomic data loading \\
Memory & $\geq$ 4~GB RAM (CPU), $\geq$ 4~GB VRAM (GPU) \\
\bottomrule
\end{tabular}
\end{table}

\section{Step-by-Step Setup}

\begin{lstlisting}[style=bashstyle,caption={Complete setup procedure}]
# Clone the repository
git clone git@github.com:kjhan0606/lumina-sn.git
cd lumina-sn

# Build CPU version
make

# (Optional) Build GPU version
make cuda

# Verify with a quick test (1000 packets, 5 iterations)
./lumina tardis_reference 1000 5

# Production run (200K packets, 20 iterations)
./lumina tardis_reference 200000 20

# Check output
head lumina_spectrum.csv
\end{lstlisting}

\section{Input Directory Structure}\index{Input Files}

LUMINA expects a reference data directory (default: \texttt{data/tardis\_reference/}) containing:

\begin{lstlisting}[style=bashstyle,caption={Required input files}]
data/tardis_reference/
  config.json                # Simulation parameters
  geometry.csv               # Shell radii and velocities
  density.csv                # Mass density per shell
  abundances.csv             # Element mass fractions
  electron_densities.csv     # Initial electron densities
  plasma_state.csv           # Initial W, T_rad per shell
  line_list.csv              # Atomic line data
  tau_sobolev.npy            # Reference optical depths
  transition_probabilities.npy
  macro_atom_data.csv
  macro_atom_references.csv
  line2macro_level_upper.npy
  levels.csv                 # Energy levels
  ionization_energies.csv
  zeta_ions.csv
  zeta_temps.csv
  zeta_data.npy
  atom_masses.csv
\end{lstlisting}

\section{Output Files}\index{Output Files}

\begin{table}[h]
\centering
\caption{Output files produced by LUMINA.}
\begin{tabular}{ll}
\toprule
\textbf{File} & \textbf{Contents} \\
\midrule
\texttt{lumina\_spectrum.csv} & $\lambda$ (\AA) vs $L_\lambda$ (erg/s/cm) \\
\texttt{lumina\_plasma\_state.csv} & Final $W$, $T_\mathrm{rad}$ per shell \\
\texttt{lumina\_tau\_validation.csv} & $\tau_\mathrm{Sob}$ at shell 0 (debug) \\
\bottomrule
\end{tabular}
\end{table}


%----------------------------------------------------------------------------------------
%	CHAPTER 11: The Ejecta Model
%----------------------------------------------------------------------------------------

\chapter{The Ejecta Model}\index{Ejecta Model}

\section{Three-Zone Composition}\index{Ejecta Model!Three-Zone}

LUMINA uses a three-zone abundance structure motivated by SN~Ia nucleosynthesis:

\begin{table}[h]
\centering
\caption{Default three-zone composition model.}
\begin{tabular}{lcccccccc}
\toprule
\textbf{Zone} & \textbf{Fe} & \textbf{Si} & \textbf{S} & \textbf{Ca} & \textbf{Co} & \textbf{Ni} & \textbf{C} & \textbf{O} \\
\midrule
Core ($v < v_\mathrm{core}$) & \textit{free} & 0.05 & 0.05 & 0.03 & 0.05 & \textit{free} & 0.02 & filler \\
Wall ($v_\mathrm{core}$--$v_\mathrm{wall}$) & \textit{free} & \textit{free} & 0.05 & 0.03 & 0.05 & \textit{free} & 0.02 & filler \\
Outer ($v > v_\mathrm{wall}$) & \textit{free} & 0.02 & 0.02 & 0.01 & 0.05 & \textit{free} & 0.02 & filler \\
\bottomrule
\end{tabular}
\end{table}

\noindent ``Filler'' means oxygen fills the remaining mass fraction to ensure $\sum X_i = 1$.

\begin{important}
Oxygen is the correct filler element for the outer zone because it has very few optical absorption lines and is essentially transparent. Using Fe/Ni/S as fillers creates massive line blanketing ($> 10^6$ active lines) that produces an artificial pseudo-photosphere (Task \#063).
\end{important}

\section{Broken Power-Law Density}\index{Ejecta Model!Density Profile}

The density profile is a broken power law:
\begin{equation}
\rho(v) = \begin{cases}
\rho_0 \left(\dfrac{v}{v_\mathrm{inner}}\right)^{n_\mathrm{inner}} & v < v_\mathrm{break} \\[10pt]
\rho_\mathrm{break} \left(\dfrac{v}{v_\mathrm{break}}\right)^{n_\mathrm{outer}} & v \geq v_\mathrm{break}
\end{cases}
\end{equation}
where $\rho_\mathrm{break} = \rho_0 (v_\mathrm{break}/v_\mathrm{inner})^{n_\mathrm{inner}}$ ensures continuity at $v_\mathrm{break}$.

Typical values: $n_\mathrm{inner} \approx -7$, $n_\mathrm{outer} \approx -10$.

\section{Physical Parameter Space}\index{Ejecta Model!Parameter Space}

LUMINA-ML (the machine learning emulator companion) uses a 15-dimensional parameter space:

\begin{table}[h]
\centering
\caption{Full 15D parameter space with ranges.}
\label{tab:params_15d}
\small
\begin{tabular}{clccl}
\toprule
\textbf{\#} & \textbf{Parameter} & \textbf{Min} & \textbf{Max} & \textbf{Description} \\
\midrule
1 & $\log L$ & 42.50 & 43.50 & Luminosity [erg/s] \\
2 & $v_\mathrm{inner}$ & 7000 & 15000 & Photosphere velocity [km/s] \\
3 & $\log\rho_0$ & $-14.0$ & $-12.3$ & Reference density [g/cm$^3$] \\
4 & $n_\mathrm{inner}$ & $-10$ & $-4$ & Inner density exponent \\
5 & $T_e/T_\mathrm{rad}$ & 0.7 & 1.0 & Temperature ratio \\
6 & $v_\mathrm{core}$ & 9000 & 17000 & Core/wall boundary [km/s] \\
7 & $v_\mathrm{wall}$ & 12000 & 24000 & Wall/outer boundary [km/s] \\
8 & $X_\mathrm{Fe,core}$ & 0.05 & 0.85 & Core iron abundance \\
9 & $X_\mathrm{Si,wall}$ & 0.05 & 0.75 & Wall silicon abundance \\
10 & $v_\mathrm{break}$ & 10000 & 22000 & Density break velocity [km/s] \\
11 & $n_\mathrm{outer}$ & $-14$ & $-4$ & Outer density exponent \\
12 & $t_\mathrm{exp}$ & 10 & 28 & Time since explosion [days] \\
13 & $X_\mathrm{Fe,wall}$ & 0.001 & 0.50 & Wall iron contamination \\
14 & $X_\mathrm{Ni}$ & 0.005 & 0.25 & Nickel abundance (all zones) \\
15 & $X_\mathrm{Fe,outer}$ & 0.001 & 0.15 & Outer iron abundance \\
\bottomrule
\end{tabular}
\end{table}


%----------------------------------------------------------------------------------------
%	CHAPTER 12: Parameter Fitting
%----------------------------------------------------------------------------------------

\chapter{Parameter Fitting}\index{Parameter Fitting}

\section{Fitting Strategy}

LUMINA includes a multi-phase parameter search framework (\texttt{scripts/fit\_parameter\_search.py}) that combines Latin Hypercube Sampling with progressive refinement.

\subsection{Phase 1: Coarse Exploration}
\begin{itemize}
\item 200 Latin Hypercube samples across full parameter space
\item 20K packets $\times$ 5 iterations (fast, $\sim$5~s per model)
\item Score by feature-weighted RMS
\item Select top-20 candidates
\end{itemize}

\subsection{Phase 2: Refinement}
\begin{itemize}
\item Top-20 candidates re-simulated with 100K packets $\times$ 10 iterations
\item Better statistics reduce Monte Carlo noise
\item Select top-3
\end{itemize}

\subsection{Phase 3: Production}
\begin{itemize}
\item Top-3 candidates with 500K packets $\times$ 20 iterations
\item Highest-fidelity spectra
\item Final selection based on composite score
\end{itemize}

\section{Objective Function}\index{Parameter Fitting!Objective Function}

The composite scoring function includes:
\begin{equation}
\text{Score} = \text{RMS}_\mathrm{spec} + 0.5\,|\Delta d_\mathrm{Si\,II}| + 0.2\,|\Delta \log v_\mathrm{Si\,II}| + 0.1\,|\Delta\lambda_\mathrm{min}|
\end{equation}
where:
\begin{itemize}
\item $\text{RMS}_\mathrm{spec}$: Spectral RMS over 5000--8000~\AA
\item $\Delta d_\mathrm{Si\,II}$: Si~\textsc{ii} 6355 absorption depth error
\item $\Delta \log v_\mathrm{Si\,II}$: Si~\textsc{ii} velocity error (log scale)
\item $\Delta\lambda_\mathrm{min}$: Si~\textsc{ii} trough wavelength error
\end{itemize}


%----------------------------------------------------------------------------------------
%	CHAPTER 13: Physical Constants
%----------------------------------------------------------------------------------------

\chapter{Physical Constants \& Reference Values}\index{Physical Constants}

\begin{table}[h]
\centering
\caption{Physical constants used in LUMINA (CGS).}
\begin{tabular}{llr}
\toprule
\textbf{Symbol} & \textbf{Description} & \textbf{Value} \\
\midrule
$c$ & Speed of light & $2.99792458 \times 10^{10}$~cm/s \\
$h$ & Planck constant & $6.62607015 \times 10^{-27}$~erg$\cdot$s \\
$k_B$ & Boltzmann constant & $1.380649 \times 10^{-16}$~erg/K \\
$\sigma_\mathrm{SB}$ & Stefan--Boltzmann & $5.670374 \times 10^{-5}$~erg/cm$^2$/s/K$^4$ \\
$\sigma_T$ & Thomson cross-section & $6.6525 \times 10^{-25}$~cm$^2$ \\
$m_e$ & Electron mass & $9.10938 \times 10^{-28}$~g \\
$e$ & Electron charge & $4.80321 \times 10^{-10}$~esu \\
$\pi e^2/(m_e c)$ & Sobolev coefficient & $2.6540 \times 10^{-2}$~cm$^2$/s \\
$T_\mathrm{rad,const}$ & $T_\mathrm{rad}$ estimator constant & $1.2523 \times 10^{-11}$~K$\cdot$s \\
\bottomrule
\end{tabular}
\end{table}


%----------------------------------------------------------------------------------------
%	BIBLIOGRAPHY
%----------------------------------------------------------------------------------------

\chapter*{Bibliography}
\addcontentsline{toc}{chapter}{\textcolor{ocre}{Bibliography}}
\printbibliography[heading=none]

%----------------------------------------------------------------------------------------
%	INDEX
%----------------------------------------------------------------------------------------

\cleardoublepage
\phantomsection
\setlength{\columnsep}{0.75cm}
\addcontentsline{toc}{chapter}{\textcolor{ocre}{Index}}
\printindex

\end{document}
